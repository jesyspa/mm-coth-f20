\documentclass{article}
\usepackage[utf8]{inputenc}
\usepackage[english]{babel}
\usepackage{amsthm}
\usepackage{amsmath}

\newtheorem{theorem}{Theorem}
\newtheorem{lemma}[theorem]{Lemma}

\newcommand{\tuple}[1]{\langle #1 \rangle}
\newcommand{\concat}{\mathbin{\widehat{}}}

\title{Computability Theory: Week 1}
\author{Anton Golov (agolov@science.ru.nl)}
\begin{document}
  \maketitle

  Exercises: 2.5.4 (6), 2.5.9

  \begin{theorem}[2.5.4 (6)]
    Given a primitive recursive function $f$, there is a primitive recursive
    function $s$ such that for all $\vec x, z$,
    \begin{equation}
      \label{eq:sum}
      s(\vec x, y) = \sum_{y \le z} f(\vec x, y).
    \end{equation}
  \end{theorem}

  \begin{proof}
    Let
    \begin{align*}
      g(\vec x) &= f(\vec x, 0)\\
      h(r, \vec x, n) &= r + f(\vec x, n+1)
    \end{align*}
    and let $s$ be defined from $g$ and $h$ by primitive recursion.

    We prove \ref{eq:sum} by induction.

    In the base case,
    \[
      s(\vec x, 0) = g(\vec x) = f(\vec x, 0) = \sum_{y \le 0} f(\vec x, y).
    \]

    Suppose now that
    \[
      s(\vec x, z) = \sum_{y \le z} f(\vec x, z)
    \]
    for some $z$.  Then
    \begin{align*}
      s(\vec x, z+1) &= h(s(\vec x, z), \vec x, z)\\
      &= s(\vec x, z) + f(\vec x, z+1)\\
      &= \sum_{y \le z} f(\vec x, y) + f(\vec x, z+1)\\
      &= \sum_{y \le z+1} f(\vec x, y).
    \end{align*}
  \end{proof}

  \begin{theorem}[2.5.9]
    Given primitive recursive $g$ and $h$, there exists a primitive recursive
    function $f$ such that
    \begin{align*}
      f(\vec x, 0) &= g(\vec x)\\
      f(\vec x, n+1) &= h(\tuple{f(\vec x, 0), \ldots, f(\vec x, n)}, \vec x, n).
    \end{align*}
  \end{theorem}

  \begin{proof}
    Let
    \begin{align*}
      g_L(\vec x) &= \tuple{g(\vec x)}\\
      h_L(r, \vec x, n) &= r \concat \tuple{h(r, \vec x, n)}.
    \end{align*}
    and define $f_L$ from $g_L$ and $h_L$ by primitive recursion. Let $f(\vec x, n) = (f_L(\vec x, n))_n$.

    For the first equation, we have
    \[
      f(\vec x, 0) = (f_L(\vec x, 0))_0
      = (g_L(\vec x))_0
      = (\tuple{g(\vec x)})_0
      = g(\vec x).
    \]

    For the second equation, we first show
    \[
      f_L(\vec x, n) = \tuple{f(\vec x, 0), \ldots, f(\vec x, n)}.
    \]
    We proceed by induction.  The base case holds by definition.
    For the inductive case, note
    \begin{align*}
      f_L(\vec x, n+1)
      &= f_L(\vec x, n) \concat \tuple{h(f_L(\vec x, n), \vec x, n)}\\
      &= \tuple{f(\vec x, 0), \ldots, f(\vec x, n)} \concat \tuple{h(f_L(\vec x, n), \vec x, n)}\\
      &= \tuple{f(\vec x, 0), \ldots, f(\vec x, n), h(f_L(\vec x, n), \vec x, n)}.
    \end{align*}

    Since $(f_L(\vec x, n+1))_{n+1} = f(\vec x, n+1)$ by definition, it follows that
    \[
      h(f_L(\vec x, n), \vec x, n) = f(\vec x, n+1).
    \]

    Now note
    \[
      f(\vec x, n+1) = h(f_L(\vec x, n), \vec x, n) = h(\tuple{f(\vec x, 0), \ldots, f(\vec x, n)}, \vec x, n)
    \]
    as required.
  \end{proof}
\end{document}
