\documentclass{article}
\usepackage[utf8]{inputenc}
\usepackage[english]{babel}
\usepackage{amsthm}
\usepackage{amsmath}
\usepackage{amsfonts}

\newtheorem{theorem}{Theorem}
\newtheorem{lemma}[theorem]{Lemma}

\newcommand{\tuple}[1]{\langle #1 \rangle}
\newcommand{\concat}{\mathbin{\widehat{}}}
\newcommand{\set}[1]{\{ #1 \}}
\newcommand{\compr}[2]{\{ #1 \,:\, #2 \}}
\newcommand{\terminates}{\!\!\downarrow}
\newcommand{\diverges}{\!\!\uparrow}
\newcommand{\njump}{\emptyset^{(n)}}
\newcommand{\njumpprime}{\emptyset^{(n)^\prime}}
\newcommand{\snjump}{\emptyset^{(n+1)}}
\DeclareMathOperator{\Fin}{Fin}
\DeclareMathOperator{\Comp}{Comp}

\title{Computability Theory: Week 9}
\author{Anton Golov (agolov@science.ru.nl)}
\begin{document}
  \maketitle

  Exercise: 5.6.4

  \begin{theorem}
    For all $n \ge 1$, $\njump \equiv_m K_n$.
  \end{theorem}

  By Proposition~4.1.5, $K_n$ is $m$-complete for $\Sigma^0_n$.  It therefore suffices to show that $\njump$ is
  also $m$-complete for $\Sigma^0_n$.  This fact is already remarked on in the proof of Proposition~5.2.1; let us make
  explicit how the relativization is done.

  \begin{lemma}[Relativization of Proposition~3.5.3]
    For all sets $A, B$, $A \in \Sigma^{0, B}_1$ iff $A \le_m B'$.
  \end{lemma}

  \begin{proof}
    Suppose $A \in \Sigma^{0, B}_1$.  There is some $R$ such that $x \in A$ iff $(\exists y) R(x, y)$.  Define $f$
    $B$-computable such that
    \[
      \varphi_{f(x)}(z) =
      \begin{cases}
        0 & \text{if $(\exists y) R(x, y)$}\\
        \uparrow & \text{otherwise}
      \end{cases}
    .\]

    Now $x \in A$ iff $\varphi^B_{f(x)}(f(x))\terminates \iff f(x) \in B'$, hence $A \le_m B'$.

    On the other hand, suppose $x \in A$ iff $f(x) \in B'$ for some computable $f$.  Let
    \[
      B_s' = \compr{x}{\varphi^B_{x, s}(x)\terminates}.
    \]

    Note that $B_s'$ is $B$-computable.  Now $A = \compr{x}{(\exists s)(f(x) \in B_s')}$, and hence $A \in \Sigma^{0, B}_1$.
  \end{proof}

  \begin{lemma}
    For all $n$, $\snjump$ is $m$-complete for $\Sigma^{0, \njump}_1$.
  \end{lemma}

  \begin{proof}
    First, note that $\snjump \in \Sigma^{0, \njump}_1$ since $\snjump = \njumpprime$ and thus $\snjump \le_m
    \njumpprime$, which sufficies by the lemma.  Now let $A \in \Sigma^{0, \njump}_1$; by Lemma~2, $A \le_m \njumpprime$,
    which is exactly what we set out to show.
  \end{proof}

  \begin{lemma}
    For all $n$, $\Sigma^{0, \njump}_1 = \Sigma^0_{(n+1)}$.
  \end{lemma}

  \begin{proof}
    This is exactly Proposition~5.2.1 (iii).
  \end{proof}

  \begin{proof}[Of Theorem~1]
    Since both $\njump$ and $K_n$ are $m$-complete for $\Sigma^0_n$, they are $m$-equivalent.
  \end{proof}

\end{document}
