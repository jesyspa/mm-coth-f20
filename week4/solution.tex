\documentclass{article}
\usepackage[utf8]{inputenc}
\usepackage[english]{babel}
\usepackage{amsthm}
\usepackage{amsmath}
\usepackage{amsfonts}

\newtheorem{theorem}{Theorem}
\newtheorem{lemma}[theorem]{Lemma}

\newcommand{\tuple}[1]{\langle #1 \rangle}
\newcommand{\concat}{\mathbin{\widehat{}}}
\newcommand{\set}[1]{\{ #1 \}}
\newcommand{\compr}[2]{\{ #1 \,:\, #2 \}}
\newcommand{\terminates}{\!\!\downarrow}
\newcommand{\diverges}{\!\!\uparrow}

\title{Computability Theory: Week 4}
\author{Anton Golov (agolov@science.ru.nl)}
\begin{document}
  \maketitle

  Exercises: 3.8.11

  \begin{theorem}[3.8.11]
    There is a reduction $\bar K \le_m \compr{e}{W_e = \emptyset}$.
  \end{theorem}

  Recall that a proof of this consists of three components:
  \begin{enumerate}
    \item A unary computable function $f$.
    \item A proof that for all $n \in \bar K$, $f(n) \in \compr{e}{W_e = \emptyset}$.  That is, for all $n$, if
    $\phi_n(n)\diverges$ then $W_{f(n)} = \emptyset$.
    \item A proof that for all $n \not \in \bar K$, $f(n) \not \in \compr{e}{W_e = \emptyset}$.  That is, for all $n$,
      if $\phi_n(n)\terminates$ then $W_{f(n)} \neq \emptyset$.
  \end{enumerate}

  \begin{proof}
    Define $f$ (using the S-m-n theorem) by
    \[
      \phi_{f(n)}(x) = \phi_n(n).
    \]

    Suppose $n \not \in K$.  Then $\phi_n(n)\diverges$ and thus $W_{f(n)} = \emptyset$.  On the other hand, if $n \in
    K$, then $\phi_n(n)\terminates$, so $W_{f(n)} = \omega \neq \emptyset$.
  \end{proof}
\end{document}
