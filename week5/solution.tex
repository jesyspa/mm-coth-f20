\documentclass{article}
\usepackage[utf8]{inputenc}
\usepackage[english]{babel}
\usepackage{amsthm}
\usepackage{amsmath}
\usepackage{amsfonts}

\newtheorem{theorem}{Theorem}
\newtheorem{lemma}[theorem]{Lemma}

\newcommand{\tuple}[1]{\langle #1 \rangle}
\newcommand{\concat}{\mathbin{\widehat{}}}
\newcommand{\set}[1]{\{ #1 \}}
\newcommand{\compr}[2]{\{ #1 \,:\, #2 \}}
\newcommand{\terminates}{\!\!\downarrow}
\newcommand{\diverges}{\!\!\uparrow}

\title{Computability Theory: Week 5}
\author{Anton Golov (agolov@science.ru.nl)}
\begin{document}
  \maketitle

  Exercises: 3.8.15

  \begin{theorem}[3.8.15 (i)]
    There is an $e$ such that $W_e = \set{e}$.
  \end{theorem}

  \begin{proof}
    Using the S-m-n theorem define $f$ by
    \[
      \phi_{f(x)}(y) =
      \begin{cases}
        0 & \text{if $y = x$}\\
        \uparrow & \text{otherwise}
      \end{cases}.
    \]

    By the recursion theorem, there is an $e \in \mathbb{N}$ such that $\phi_{f(e)} = \phi_e$.
    This $e$ satisfies the condition of the theorem, since
    \[
      y \in W_e \iff y \in W_{f(e)} \iff y = e.
    \]
  \end{proof}

  \begin{theorem}[3.8.15 (ii)]
    There is an $e$ such that for every $x$, $\phi_e(x)\terminates \iff \phi_{2e}(x)\terminates$.
  \end{theorem}

  \begin{proof}
    Since $x \mapsto 2x$ is a computable function, by the recursion theorem, there is an
    $e \in \mathbb{N}$ such that $\phi_{2e} = \phi_e$.
  \end{proof}

  \begin{theorem}[3.8.15 (iii)]
    There is an $e$ such that $W_e = D_e$.
  \end{theorem}

  \begin{proof}
    By exercise 2.5.16 there is a computable function $f$ such that $W_{f(n)} = D_n$.  By the
    recursion theorem, there is an $e \in \mathbb{N}$ such that $\phi_{f(e)} = \phi_e$.
    This $e$ satisfies the condition of the theorem, since
    \[
      W_e = W_{f(e)} = D_e.
    \]
  \end{proof}
\end{document}
