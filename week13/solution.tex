\documentclass{article}
\usepackage[utf8]{inputenc}
\usepackage[english]{babel}
\usepackage{amsthm}
\usepackage{amsmath}
\usepackage{amsfonts}

\newtheorem{theorem}{Theorem}
\newtheorem{lemma}[theorem]{Lemma}

\newcommand{\tuple}[1]{\langle #1 \rangle}
\newcommand{\concat}{\mathbin{\widehat{}}}
\newcommand{\set}[1]{\{ #1 \}}
\newcommand{\compr}[2]{\{ #1 \,:\, #2 \}}
\newcommand{\terminates}{\!\!\downarrow}
\newcommand{\diverges}{\!\!\uparrow}
\newcommand{\njump}{\emptyset^{(n)}}
\newcommand{\njumpprime}{\emptyset^{(n)^\prime}}
\newcommand{\snjump}{\emptyset^{(n+1)}}
\newcommand{\PiOne}{\Pi^0_1}
\newcommand{\Cons}{\widehat{\,\,}}
\DeclareMathOperator{\Fin}{Fin}
\DeclareMathOperator{\Tot}{Tot}
\DeclareMathOperator{\Comp}{Comp}
\DeclareMathOperator{\dom}{dom}

\title{Computability Theory: Week 13}
\author{Anton Golov (agolov@science.ru.nl)}
\begin{document}
  \maketitle

  Exercise: 7.4.5.

  \begin{theorem}
    For any $k$, the set of $k$-random strings is $\Pi^0_1$.
  \end{theorem}

  \begin{proof}
    Fix $k$.  Recall that a string $\sigma$ is $k$-random if \[ C(\sigma) = \min\compr{|\tau|}{U(\tau) = \sigma} \ge
    |\sigma| - k. \]

    Negating the condition, a string is non-$k$-random if there exists a $\tau$ such that $U(\tau) =
    \sigma$ and $|\tau| < |\sigma| - k$.

    Note now that given $\sigma$, if there is such a $\tau$ then we can find it: compute all $U(\tau)$ in parallel and
    halt if you find one that satisfies the requirements.  It follows that the set of non-$k$-random strings is
    $\Sigma^0_1$, and its complement is therefore $\Pi^0_1$
  \end{proof}
\end{document}
