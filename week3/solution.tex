\documentclass{article}
\usepackage[utf8]{inputenc}
\usepackage[english]{babel}
\usepackage{amsthm}
\usepackage{amsmath}
\usepackage{amsfonts}

\newtheorem{theorem}{Theorem}
\newtheorem{lemma}[theorem]{Lemma}

\newcommand{\tuple}[1]{\langle #1 \rangle}
\newcommand{\concat}{\mathbin{\widehat{}}}

\title{Computability Theory: Week A}
\author{Anton Golov (agolov@science.ru.nl)}
\begin{document}
  \maketitle

  Exercises: 3.8.5

  \begin{theorem}[3.8.5 (Proposition 3.2.6)]
    A set $A$ is computable if and only if both $A$ ands its complement $\bar A$ are c.e.
  \end{theorem}

  \begin{proof}
    Suppose $A$ is computable.  Then $A$ is c.e.\@and $\bar A$ is also computable, so $\bar A$ is also c.e.

    Suppose $A$ and $\bar A$ are c.e. and let $e, d \in \mathbb{N}$ such that $A = W_e$ and $\bar A = W_d$.
    Define
    \begin{align*}
      t(x) &= (\mu s)[x \in W_{e, s} \cup W_{d, s}]\\
      g(x) &= \chi_{W_{e, t(x)}}(x).
    \end{align*}

    Note that $t$ and $g$ are total: for every $x$, since $x \in A$ or $x \in \bar A$, there is some $s$ such that
    $x \in W_{e,s}$ or $x \in W_{d,s}$.  Thus $t$ is total and since $W_{e,s}$ is uniformly computable in
    $s \in \mathbb{N}$, so is $g$.

    Now suppose $x \in A$.  Since $x \not \in W_d$, $x \in W_{e,t(x)}$ and thus $g(x) = 1$.  On the other hand, suppose
    $x \in \bar A$.  Since $x \not \in W_e$ and $W_{e, s} \subseteq W_e$ for all $s \in \mathbb{N}$, it follows that
    $g(x) = 0$.  Thus $g = \chi_A$ and so $A$ is computable.
  \end{proof}
\end{document}
